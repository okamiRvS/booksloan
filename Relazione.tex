\documentclass[a4paper,10pt]{article}
\usepackage[hidelinks]{hyperref}
\usepackage{float}
\usepackage{graphicx}
\usepackage{listings}
\usepackage[utf8]{inputenc}

%%% Title
\title{Processo e Sviluppo Software: Assignment 3} 
\author{Ivo Junior Bettini - 806878, Umberto Cocca - 807191, \\Silvia Traversa - 816435\\
\href{https://gitlab.com/s.traversa/2019_assignment3_booksloan}{GitLab repository}}
\date{}

\begin{document}

\maketitle 

\section*{Applicazione}
L'applicazione BooksLoan permette agli utenti di visualizzare il catalogo di una biblioteca. Essi possono visualizzare sia le copie disponibili, ed eventualmente richiederne una in prestito, sia quelle non disponibili e prenotarle per quando ritorneranno presso la biblioteca.
È inoltre possibile visualizzare le informazioni relative a ogni singolo libro, quali l'autore o la presenza di sequel.\\ 

\noindent Gli amministratori, che si dividono in amministratori con contratto a tempo indeterminato e amministratori con contratto a tempo determinato, differiscono per la possibilità di eliminare i libri. Possono entrambi interagire con un'area CMS eseguendo il CRUD delle principali entità. 

\section*{Esecuzione dell'applicazione}
\subsection*{Preparazione Database}
Prima di eseguire l'applicazione, è necessario importare i dati e la struttura del database, creato con MySQL.\\\\
Per effettuare questa operazione bisogna effettuare su MySQLWorkbench un Data Import del file \textit{DumpFinale} presente nella cartella \textit{dumps}. Successivamente, per permettere la connessione al database, è richiesta la modifica del file \textit{src/main/resources/application.properties} inserendo le proprie credenziali MySQL nei campi spring.datasource.username e spring.datasource.password, come illustrato nella figura.\\
\begin{figure}[H]
	\centering
	\includegraphics[width=1\linewidth]{images/properties}
\end{figure}
\newpage

\subsection*{Avvio Applicazione}

Per far partire l'applicazione, dopo aver clonato il repository, eseguire le seguenti righe di comando:\\

\begin{lstlisting}[language=bash]
	mvnw clean package spring-boot:repackage
\end{lstlisting}

\begin{lstlisting}[language=bash]
	java target/BooksLoan-1.jar
\end{lstlisting}

\noindent \\L'applicazione sarà disponibile all'indirizzo \href{http://localhost:8080}{http://localhost:8080}.\\

\noindent Per accedere alle funzionalità dell'applicazione è rischiesto un login, che è possibile effettuare con tre tipi diversi di account, rappresentativi di categorie:

\begin{itemize}
	\item \textbf{amministratore con contratto a tempo indeterminato:}
\begin{lstlisting}[language=bash]
    username: 123 password: 1234
\end{lstlisting}
	\item \textbf{amministratore con contratto a tempo determinato:}
\begin{lstlisting}[language=bash]
    username: 456 password: 4567
\end{lstlisting}
	\item \textbf{cliente:}
\begin{lstlisting}[language=bash]
    username: 321 password: 1234
\end{lstlisting}
\end{itemize}

\section*{Schema E-R}

\begin{figure}[H]
	\centering
	\includegraphics[width=0.8\linewidth]{images/ERdiagram}
	\caption[Schema ER]{Schema ER}
	\label{fig:re}
\end{figure}

Le entità utilizzate nel nostro progetto sono così descritte:

\begin{itemize}
	\item Utente: descrive la persona registrata e che utilizza il sito. Gli attributi n\_tessera e password rappresentano le credenziali d'accesso al sito, mentre l'attributo ruolo determina se l'utente è un amministratore o un utente.
	\item Libro: elemento che compone la nostra libreria online.
	\item Copia: insieme di diverse edizioni di un libro che possiedono l'ISBN come chiave primaria. Una copia può trovarsi in due stati, disponibile o non disponibile (in quest'ultimo caso non può essere presa in prestito).
	\item Autore: colui che ha scritto uno o più libri.
\end{itemize}

Esse sono messe in relazione da:
\begin{itemize}
	\item Prestito: entità che relaziona utente con copia. Un utente può prenotare più copie dello stesso libro.
	\item Sequel: entità che mette in relazione un libro con se stesso. Un libro può possedere zero o più sequel.
	\item Scritto: entità che mette in relazione autore con libro. Un libro può essere scritto da più autori e un autore può scrivere più libri.
\end{itemize}

\section*{BooksLoan}
Per poter accedere all’applicazione bisogna obbligatoriamente essere registrati. Attraverso l'utilizzo della dipendenza Spring Security viene controllato l'accesso, impedendone la navigazione tra le pagine se non si è registrati e gestendone i livelli d'accesso. Nello specifico un cliente non può accedere alle pagine di un amministratore.\\

\noindent L'amministratore con un contratto a tempo indeterminato, una volta che ha eseguito il login, ha la possibilità di effettuare ogni tipo di operazione: può eliminare, aggiungere, modificare libri e impostare le copie, gli autori e i sequel di un libro.
Un amministratore con contratto a tempo determinato, invece, non può cancellare i libri e le sue relative copie.\\

\noindent Il cliente, una volta effettuato il login, ha a disposizione l'elenco dei libri del catalogo della biblioteca. Questi sono riportati in una tabella in cui è possibile filtrare cercando per chiave. Il filtraggio è stato eseguito utilizzando jQuery. Inoltre, il cliente può visualizzare se è presente una copia disponibile di un determinato libro ed eventualmente prenderla in prestito, può visualizzare i suoi prestiti in corso e restituire una copia. La restituzione rende disponibile la copia stessa e quindi accessibile da altri clienti per la prenotazione.\\

\end{document}